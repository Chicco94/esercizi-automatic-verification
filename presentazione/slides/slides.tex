\documentclass[table]{beamer}

\usepackage{pgfpages}
\setbeameroption{show notes on second screen}

\bibliographystyle{plain}

\usepackage[utf8]{inputenc}
\usepackage[english]{babel}

\usepackage[default]{droidserif}
\usepackage[T1]{fontenc}
\usepackage{ragged2e}
\usepackage{amsmath,adjustbox}
\usepackage{mathtools}
\usepackage{lmodern} % https://tex.stackexchange.com/questions/58087/how-to-remove-the-warnings-font-shape-ot1-cmss-m-n-in-size-4-not-available
\apptocmd{\frame}{}{\justifying}{}

\usepackage{changepage}
\usepackage{graphics}
\usepackage{tikz}

\usepackage{showframe}


% per immagini
\usepackage{graphicx}
\graphicspath{ {./images/} }

% per testo colotaro
\usepackage{xcolor}

% colori utilizzati
\definecolor{marrone}{RGB}{92,64,60}
\definecolor{arancio}{RGB}{255,102,0}
\definecolor{azzurro}{RGB}{119,154,171}
\definecolor{azzurro_chiaro}{RGB}{198,214,220}
\definecolor{bianco}{RGB}{255,255,255}


%--------------------------------------------------------------------------------
%						BLOCCO PER TEOREMI E DEFINIZIONI
%--------------------------------------------------------------------------------
\usepackage{amsthm}
\usepackage{thmtools}
\usepackage{thmbox}

\makeatletter
\def\thmbox@color{black}
\define@key{thmbox}{color}{\def\thmbox@color{#1}}
\def\thmbox@head#1{%
  \par\noindent\vbox{%
    \setbox\thmbox@box@=\hbox{%
      \vrule width 0mm height 0mm depth \thmbox@vskip%
      #1}%
    \copy\thmbox@box@%
    \ifthmbox@underline%
       \color@begingroup\color{\thmbox@color}\hrule width \wd\thmbox@box@ height \thmbox@thickness\color@endgroup%
    \fi}%
  \hrule height 0mm\relax}
\def\thmbox@put#1{
  \vskip\z@%
  \noindent%
  \hbox{%
    {\dimen0=\thmbox@leftmargin%
     \advance\dimen0-\thmbox@hskip%
     \advance\dimen0-\thmbox@thickness%
     \hskip\dimen0}%
    \color@begingroup\color{\thmbox@color}\vrule width \thmbox@thickness\color@endgroup%
    \hskip\thmbox@hskip%
    \box#1%
    \ifx\thmbox@style L%
      \hskip\thmbox@hskip%
      \color@begingroup\color{\thmbox@color}\vrule width \thmbox@thickness\color@endgroup%
    \fi}%
  \par\nobreak}
\def\thmbox@tail{%
  \hrule height 0mm%
  \ifx\thmbox@style M%
    \thmbox@dim=1cm%
  \else\ifx\thmbox@style L%
    \thmbox@dim=\hsize%
    \advance\thmbox@dim-\thmbox@leftmargin%
    \advance\thmbox@dim-\thmbox@rightmargin%
    \advance\thmbox@dim2\thmbox@hskip%
    \advance\thmbox@dim2\thmbox@thickness%
  \fi\fi%
  \noindent%
  {\dimen0=\thmbox@leftmargin%
   \advance\dimen0-\thmbox@hskip%
   \advance\dimen0-\thmbox@thickness%
   \hskip\dimen0}%
  \color@begingroup\color{\thmbox@color}\vrule width \thmbox@dim height \thmbox@thickness\color@endgroup%
  \par}
\makeatother

%\declaretheorem[shaded={rulecolor=azzurro,rulewidth=2pt, bgcolor={rgb}{1,1,1}}]{defn}{Definition}
%\declaretheorem[thmbox=M]{Definition}
\newtheorem[style=L, thickness=1pt,color=azzurro]{thm}{Theorem}
\newtheorem[style=L, thickness=1pt,color=arancio]{dfn}{Definition}
%--------------------------------------------------------------------------------
%						BLOCCO PER TEOREMI E DEFINIZIONI
%--------------------------------------------------------------------------------



%--------------------------------------------------------------------------------
%						BLOCCO PER PRESENTAZIONI UNIUD
%--------------------------------------------------------------------------------
\setbeamercolor{marrone}{fg=white, bg=marrone}
\setbeamercolor{arancio}{fg=white, bg=arancio}
\setbeamercolor{azzurro}{fg=white, bg=azzurro}

\setbeamertemplate{headline}{%
	\leavevmode%
	\begin{beamercolorbox}[sep=0.15cm,ht=1cm,wd=1.05cm]{marrone}%
		\includegraphics[width=.75cm]{UdineLogoFull}
	\end{beamercolorbox}%
	\begin{beamercolorbox}[sep=0.15cm,ht=1cm,wd=.4\paperwidth]{arancio}%
		\textbf{UNIVERSITÀ \linebreak
		DEGLI STUDI \linebreak
		DI UDINE}
	\end{beamercolorbox}%
	\begin{beamercolorbox}[ht=1cm,wd=.6\paperwidth]{azzurro}%
	\end{beamercolorbox}%
}

\setbeamertemplate{frametitle}{
	\begin{flushright}
		\vspace*{-1.25cm}\color{bianco}
		\insertframetitle
	\end{flushright}
}

\logo{\includegraphics[scale=.5]{UdineLogo}\hspace*{.65\paperwidth}}
%--------------------------------------------------------------------------------
%					FINE BLOCCO PER PRESENTAZIONI UNIUD
%--------------------------------------------------------------------------------


%Information to be included in the title page:
\title[Automatic Verification] %optional
{On the Synthesis of Discrete Controllers for Timed Systems \cite{first_article}}

\subtitle{An Extended Abstract}

\author[Enrico] % (optional, for multiple authors)
{E. Cominato 137396\inst{1}}

\institute % (optional)
{
	\inst{1}%
	Dipartimento di Scienze Matematiche, Informatiche e Fisiche\\
	Università degli studi di Udine
}

\date[VLC 2013] % (optional)
{}

\newcommand{\Plant}{\ensuremath{\mathcal{P}=(Q,\Sigma_c,\delta, q_0)}}
\newcommand{\Controller}{\ensuremath{C:Q^+\longmapsto \Sigma_c}}
\newcommand{\AccpCond}{\ensuremath{\{(F,\square),(F,\Diamond ),(F,\Diamond\square),(F,\square\Diamond),(\mathcal{F},\mathcal{R}_n)\}}}
\newcommand{\Synth}{\ensuremath{\textbf{Synth}(\mathcal{P},\Omega)}}
\newcommand{\conf}{\ensuremath{(q,x)}}
\newcommand{\confP}{\ensuremath{(q',x')}}
\newcommand{\confH}{\ensuremath{(\hat{q},\hat{x})}}
\newcommand{\confPair}{\ensuremath{((q,x),(q',x'))}}

\begin{document}

\frame{\titlepage}

\section{Introduction}
\subsection{Abstract}
\begin{frame}
This paper presents algorithms for the automatic synthesis of the real time controllers by finding a winning strategy for certain games defined by the timed automata of Alur and Dill.
\end{frame}

\subsection{The problem}
\begin{frame}
Consider a dynamical system P, whose presentation descri\-bes all its possible behaviours.
A subset of the plant's be\-hav\-iours, satisfying some criterion is defined as good or acceptable.

\medskip

A controller C is another system which can interact with P in a certain manner by observing the state of P and by issuing control actions that influence the behaviour of P.

\note{
	\begin{enumerate}
		\item Kitchen robot
		\item Selfdrived metro
	\end{enumerate}
}
\end{frame}

\begin{frame}
The synthesis problem is then, to find out whether, for a given P, there exists a realizable controller C
such that their interaction will produce only good behaviours.
\note{
	Let's see some definition before dive in into the theorem and its proofs
}
\end{frame}

\section{Discrete Case}
\subsection{Initial Definitions}
\begin{frame}
	\begin{dfn}[Plant]
		A plant automaton is a tuple $\Plant$ where
		\begin{itemize}
			\item $Q$ is a finite set of states,
			\item $\Sigma_c$ is a set of controller commands,
			\item $\delta:Q \times \Sigma_c \longmapsto 2^Q$ is the transition function
			\item $q_0 \in Q$ is an initial state
		\end{itemize}
	\end{dfn}

	\note{
		For each controller command $\sigma \in \Sigma_c$ at some state $q \in Q$ there are several possible consequences denoted by $\delta(q,\sigma)$.
		
		\medskip

		Unlike other formulation of 2-person games, where there is an explicit description of the transition function of both players, here we represent the response of the environment as a non-deterministic choice among the transitions labeled by the same $\sigma$.
	}
\end{frame}

\begin{frame}
	\begin{dfn}[Controllers]
		A controller for a plant specified by $\Plant$ 
		is a function $\Controller$. A simple controller is a controller that 
		can be written as a function $C:Q \longmapsto \Sigma_c$.
	\end{dfn}
	\note{
		We are interested in the simpler cases of controllers 
		that base their decisions on a finite memory.
	}
\end{frame}

\begin{frame}
	\begin{dfn}[Trajectories]
		Let $\mathcal{P}$ be a plant and let $\Controller$ be a controller. 
		An infinite sequence of states $\alpha:q[0],q[1],\ldots$ such that
		 $q[0]=q_0$ is called a trajectory of $\mathcal{P}$ if 
		$$q[i+1] \in \bigcup_{\sigma \in \Sigma_c}\delta(q[i],\sigma)$$
		and a C-trajectory if $q[i+1] \in \delta(q[i],C[\alpha[0..i]])$ for every $i\geq 0$.
		The corresponding sets of trajectories are denoted by $L(\mathcal{P})$ and $L_C(\mathcal{P})$.
	\end{dfn}
\end{frame}

\begin{frame}
	For every infinite trajectory $\alpha \in L(\mathcal{P})$:
	\begin{itemize}
		\item $Vis(\alpha)$ denote the set of all states appearing in $\alpha$
		\item $Inf(\alpha)$ denote the set of all states appearing in $\alpha$ infinitely many times
	\end{itemize}
\end{frame}

\begin{frame}
	\begin{dfn}[Acceptance Condition]
		Let $\Plant$ be a plant. An acceptance condition for $\mathcal{P}$ is
		$$ \Omega \in \AccpCond$$
		where $\mathcal{F} = \{(F_i,G_i)\}^{n}_{i=1}$ and $F, F_i$ and $G_i$ 
		are certain subsets of $Q$ referred as the good states. The set of sequences of $\mathcal{P}$ that are accepted
		accordig to $\Omega$ is defined as follows:
		\begin{adjustwidth}{-1cm}{-1cm}
		\begin{table}[]
			\begin{tabular}{ll}
			\hline
			\rowcolor{azzurro_chiaro}
			$L(\mathcal{P},F,\square)$ & $\{\alpha \in L(\mathcal{P}):Vis(\alpha)\subseteq F\}$ \\ 
			$L(\mathcal{P},F,\Diamond)$ & $\{\alpha \in L(\mathcal{P}):Vis(\alpha)\cap F \neq \emptyset\}$ \\ 
			\rowcolor{azzurro_chiaro}
			$L(\mathcal{P},F,\Diamond\square)$ & $\{\alpha \in L(\mathcal{P}):Inf(\alpha)\subseteq F\}$ \\ 
			$L(\mathcal{P},F,\square\Diamond)$ & $\{\alpha \in L(\mathcal{P}):Inf(\alpha)\cap F \neq \emptyset\}$\\ 
			\rowcolor{azzurro_chiaro}
			$L(\mathcal{P},\mathcal{F},\mathcal{R}_n)$ & \shortstack{
				$\{\alpha \in L(\mathcal{P}): \exists i \alpha \in$ \\
				$L(\mathcal{P},F_i,\square\Diamond) \cap L(\mathcal{P},G_i,\Diamond\square)\}$
			} \\ 
			\hline
			\end{tabular}
		\end{table}
		\end{adjustwidth}
	\end{dfn}
	\note[item]{$\alpha$ always remains in $F$}
	\note[item]{$\alpha$ eventually visits $F$}
	\note[item]{$\alpha$ eventually remains in $F$}
	\note[item]{$\alpha$ visits $F$ infinitely often }
	\note[item]{$\alpha$ visits $F_i$ infinitely often and eventually stays in $G_i$ }
\end{frame}

\begin{frame}
	\begin{dfn}[Controller Synthesis Problem]
		For a plant $\mathcal{P}$ and an acceptance condition $\Omega$, the problem $\Synth$ is:
		Find a controller C such that $L_C(\mathcal{P})\subseteq L(\mathcal{P},\Omega)$ or otherwise show that such a controller does not exists.
	\end{dfn}
\end{frame}

\subsection{Controllable Predecessors}
\begin{frame}
	\begin{dfn}[Controllable Predecessors]
		Let $\mathcal{P}=(Q,\Sigma_c,\delta, q)$ be a plant and a set of states $P \subseteq Q$. 
		The controllable predecessors of $P$ is the set of states from which the controller 
		can "force" the plant into $P$ in one step:
		$$ \{q : \exists \sigma \in \Sigma_c \;  \delta(q,\sigma) \subseteq P\} $$
	\end{dfn}
	We define a function $\pi:2^Q \longmapsto 2^Q$, mapping a set of states $P \subseteq Q$ into 
	the set of its Controllable predecessors:
	$$ \pi(P) = \{q : \exists \sigma \in \Sigma_c \;  \delta(q,\sigma) \subseteq P\} $$
\end{frame}

\begin{frame}
	\begin{center}
		\begin{tikzpicture}[shorten >=1pt,node distance=2.5cm,on grid,auto] 
			\node[state] (q_0)					{};
			\node[state] (q_1)	[right=of q_0]	{};
			\node[state] (q_2)	[right=of q_1]	{};

			% stati predecessori
			\node[state] (p_0)	[above=of q_0]	{};
			\node[state] (p_1)	[above=of q_1]	{};
			\node[state] (p_2)	[above=of q_2]	{};
			\node[state] (p_3)	[right=of p_2]	{};
			
			% raggruppamenti
			\begin{scope}[on background layer]
				\node[fit=(q_0) (q_1) (q_2), ellipse, 
				draw=arancio,fill=arancio!30, fill opacity=0.4, label=below:$P$] (P) {};
				\node[fit=(p_1) (p_2), ellipse, 
				draw=azzurro, fill=azzurro!30, fill opacity=0.4, label=above:$\pi(P)$] (predP) {};
			\end{scope}

			\path[->] 
				(p_0) edge  node {$\sigma$} (q_0)
				(p_0) edge  node {$\sigma$} (p_1)
				(p_1) edge  node {$\sigma$} (q_0)
					  edge  node {$\sigma$} (q_1)
				(p_2) edge  node {$\sigma$} (q_2)
				(p_3) edge  node {$\sigma$} (p_2);
		\end{tikzpicture}
	\end{center}
	\note{
		the first one is not a controllable predecessor because it can have a bad consequence
	}
\end{frame}

\subsection{Theorem}
\begin{frame}
	\begin{thm}
		For every $\Omega \in \AccpCond$, the problem $\Synth$ is solvable. 
		Moreover, if $(\mathcal{P},\Omega)$ is controllable then 
		it is controllable by a simple controller.
	\end{thm}
	\textbf{Sketch of Proof}

	\medskip
	
	For a plant $\Plant$ and an acceptance condition $\Omega$, we denote $W \subseteq Q$ as the set
	of winning states, namely, the set of states from which a controller can enforce good behaviors according to $\Omega$.
\end{frame}

\begin{frame}
	We can characterize this states by the following fixed-point expressions:
	\begin{itemize}
		\item[$\square$] $\nu W \big( F \cap \pi(W)\big)$ 
		\item[$\Diamond$] $\mu W\big(F \cup \pi(W)\big)$ 
		\item[$\Diamond\square$] $\mu W \nu H\Big(\pi(H) \cap \big(F \cup \pi(W)\big)\Big)$
		\item[$\square\Diamond$] $\nu W \mu H\Big(\pi(H) \cup \big(F \cap \pi(W)\big)\Big)$
		\item[$\mathcal{R}_1$] $\mu W \Bigg\{\pi(W) \cap \nu Y \mu H . W \cup G \cap \Big(\pi(H) \cup \big(F \cap \pi(Y)\big)\Big)\Bigg\}$
	\end{itemize}
	Then the plant is controllable iff $q_0 \in W$
	\note{
		$\nu$ greatest \\
		$\mu$ least 
	}
\end{frame}

\begin{frame}
	Let see in more details how this works. Consider the case $\Diamond$:
	\begin{table}[]
		\begin{tabular}{lll}
		$\begin{aligned}
			&W_0 := \emptyset \\
			&\textrm{\textbf{for} } i:=0,1,\ldots \textrm{ \textbf{repeat}} \\
			& \quad W_{i+1} := F \cup \pi(W_i) \\
			&\textrm{\textbf{until} } W_{i+1} = W_i\\
		\end{aligned}$
		& $\quad$ &
		$\begin{aligned}
			&W_0 := \emptyset \\
			&W_1 := F \cup \pi(W_0) = F \cup \pi(\emptyset) = F \\
			&W_2 := F \cup \pi(W_1) = F \cup \pi(F)\\
			& \ldots \\
		\end{aligned}$ \\
		&&\\% empty line
		\multicolumn{3}{l}{finally: $W_n := F \cup \pi(W_{n-1}) = F \cup \pi(F \cup \pi(\ldots(F \cup \pi(F))))$}
		\end{tabular}
	\end{table}

	\note{
		chain of operations
	}
\end{frame}


\begin{frame}
	\begin{center}
		\begin{tikzpicture}[shorten >=1pt,node distance=2cm,on grid,auto] 
			\node[state,initial] (q_0)			{$q_0$};
			\node[state] (q_1)	[below=of q_0]	{$q_1$};
			\node[state] (q_2)	[right=of q_0]	{$q_2$};
			\node[state] (q_3)	[right=of q_1]	{$q_3$};
			\node[state] (q_4)	[right=of q_2]	{$q_4$};
			\node[state] (q_5)	[right=of q_3]	{$q_5$};
			\node[state,accepting] (q_6)	[right=of q_4]	{$q_6$};
		
			% raggruppamenti
			\only<2>{
				\begin{scope}[on background layer]
					\node[fit=(q_6), ellipse, 
					draw=arancio,fill=arancio!30, fill opacity=0.4, label=below:$W_1$] ($W_1$) {};
				\end{scope}	
			}

			\only<3>{
				\begin{scope}[on background layer]
					\node[fit=(q_6) (q_4), newellipse, xscale=0.75,yscale=1, 
					draw=arancio,fill=arancio!30, fill opacity=0.4, label=below:$W_2$] ($W_2$) {};
				\end{scope}	
			}

			\only<4>{
				\begin{scope}[on background layer]
					\node[fit=(q_6) (q_4) (q_2), newellipse, xscale=0.75,yscale=1, 
					draw=arancio,fill=arancio!30, fill opacity=0.4, label=below:$W_3$] ($W_3$) {};
				\end{scope}	
			}

			\only<5>{
				\begin{scope}[on background layer]
					\node[fit=(q_6) (q_4) (q_2) (q_0), newellipse, xscale=0.75,yscale=1, 
					draw=arancio,fill=arancio!30, fill opacity=0.4, label=below:$W_4$] ($W_4$) {};
				\end{scope}	
			}

			\path[->]
			(q_0) edge node {} (q_1)
				  edge node {} (q_2)
			(q_1) edge node {} (q_3)
			(q_2) edge node {} (q_3)
				  edge node {} (q_4)
			(q_3) edge node {} (q_5)
			(q_4) edge node {} (q_5)
				  edge node {} (q_6)
			(q_5) edge [bend left] node {} (q_1)
			(q_6) edge [bend right] node {} (q_0);
			
		\end{tikzpicture}
	\end{center}
	\note{concluding the proof}
\end{frame}


\subsection{Conclusions}
\begin{frame}
	\justify
	In the process of calculating $W_{i+1}$, whenever we add a state $q$ to $W_i$, 
	there must be at least one action $\sigma \in \Sigma_c$ such that $\delta(q,\sigma) \subseteq W_i$.
	
	\medskip	
	
	So we define the controller at $q$ as $C(q)=\sigma$.
	
	\medskip
	
	When the process terminates, the controller is synthesized for all the winning states. $\qed$
		
	\medskip
	
	It can be seen that if the process fails, that is $q_0 \not\in W$, then for every controller
	command there is a possibly bad consequence that will put the system outside $F$, 
	and no controller, even an infinite state one, can prevent this.

	\note{the non determinism, the environment, is playing against us}
\end{frame}

\section{Real Time Case}
\subsection{Initial Definitions}
\begin{frame}
	Timed automata are automata equipped with clocks whose values grow continuously.

	\medskip 

	Let $T$ denote $\mathbb{R}^+$ and let $X=T^d$ (the clock space).
	
	\medskip
	
	The elements of $X$ are $x=(x_1,\ldots,x_d)$ and the d-dimensional unit vector is $\textbf{1}=(1,\ldots,1)$ 
	
	\medskip
	
	\begin{dfn}[Reset functions]
		Let $F(X)$ denote the class of functions $f:X \mapsto X$ that can be written in the form $f(x_1,\ldots,x_d)=(f_1,\ldots,f_d)$ where each $f_i$ is either $x_i$ or $0$.
	\end{dfn}
	
	\note{
		The clocks interact with the transitions by participating in preconditions 
		(guards) for certain transitions and they are possibly reset when some
		transitions are taken.\\

		Since $X$ is infinite and non-countable, we need a language to express certain subsets of $X$ as well as operations on these subsets.
	}
\end{frame}

\begin{frame}
	\begin{dfn}[$k$ polyhedral sets]
		Let $k$ be a positive integer constant. We associate with $k$ three subsets of $2^X$:
		\begin{itemize}
			\item $\mathcal{H}_k$: the set of half-spaces consisting of all sets having one of the following forms
				\begin{itemize}
					\item $X$
					\item $\emptyset$
					\item $\{x \in X:x_i \# c\}$
					\item $\{x \in X:x_i-x_j \# c\}$
				\end{itemize}
				for some $\# \in \{<,\leq,>,\geq\}$ and $c \in \{0,\ldots,k\}$
			\item $\mathcal{H}^\cap_k$: the set of convex sets consisting of intersections of elements of $\mathcal{H}_k$
			\item $\mathcal{H}^*_k$: the set of k-polyhedral sets containing all sets obtained from $\mathcal{H}_k$ via union intersection and complementation
		\end{itemize}
	\end{dfn}
	\note{
		$\#$ equals hash \\
		For every $k, \mathcal{H}^*_k$ has a finite number of elements, each of which can be written as a finite union of convex sets.\\
		They are usually called $regions$
	}
\end{frame}

\begin{frame}
	\begin{dfn}[Timed Automata]
		A timed automaton is a tuple $\mathcal{T}=\left( Q,X,\Sigma,I,R,q_0\right)$ consisting of:
		\begin{itemize}
			\item $Q$ a finite set of discrete states
			\item $X$ a clock domain $X=(\mathbb{R}^+)^d$ for some $d>0$
			\item $\Sigma=\Sigma_c \cup \{e\}$ an input alphabet (including a single environment action $e$)
			\item $I:Q \mapsto \mathcal{H}^\cap_k$ as the state invariant function
			\item $R \subseteq Q \times \Sigma \times \mathcal{H}^\cap_k \times F(X) \times Q$ is
			 a set of transition relations each of the form $\langle q,\sigma,g,f,q'\rangle$ where:
			\begin{itemize}
				\item $q,q' in Q$ are states
				\item $\sigma \in \Sigma$ is a command
				\item $g \in \mathcal{H}^\cap_k$ is a guard condition
				\item $f \in F(X)$ is a reset function
			\end{itemize}
		\end{itemize}
	\end{dfn}\cite{alur1999timed}
\end{frame}

\begin{frame}
	A \emph{configuration} of $\mathcal{T}$ is a pair $(q,x) \in Q \times X$ denoting a discrete
	state and the values of the clocks.

	\medskip

	Without loss of generality, we assume that for every $q \in Q$ and for every $x \in X$ there exists $t \in T$ such that $x+\textbf{1}t \not \in I_q$.
	\note{
		That is, the automaton cannot stay in any of its discrete states forever.

		$x+\textbf{1}t = (x_1,\ldots,x_n)+(1,\ldots,1)t=(x_1+t,\ldots,x_n+t)$
		The time has the same pace in all clocks
	}
\end{frame}
\begin{frame}
	\begin{dfn}[Steps and Trajectories]
		A step of $\mathcal{T}$ is a pair of configurations $\confPair$ such that either:
		\begin{itemize}
			\item $q=q'$ and for some $t \in T, x'=x+\textbf{1}t, x \in I_q$ and $x' \in I_q$. In this case we say that $(q',x')$ is a \emph{t-successor} of $(q,x)$ and that $\confPair$ is a \emph{t-step}.
			\item There is some $r = \langle q,\sigma,g,f,q'\rangle \in R$ such that $x \in g$ and $x' = f(x)$. In this case we say that $(q',x')$ is a \emph{$\sigma$-successor} of $(q,x)$ and that $\confPair$ is a \emph{$\sigma$-step} 
		\end{itemize}
		
		\medskip
		
		A trajectory of $\mathcal{T}$ is a sequence $ \beta = (q[0],x[0]),(q[1],x[1]),\ldots $ of configurations such that for every $i$, $((q[i],x[i]),(q[i+1],x[i+1]))$ is a step.
	\end{dfn}
	\note{
		$\sigma$-steps includes the environment steps
	}
\end{frame}

\begin{frame}
	We denote the set of all trajectories that $\mathcal{T}$ can generate by $L(\mathcal{T})$.

	\medskip

	Given a trajectory $\beta$ we can define $Vis(\beta)$ and $Inf(\beta)$ as in the discrete 
	case by referring to the projection of $\beta$ on $Q$ and use $L(\mathcal{T},\Omega)$ to 
	denote acceptable trajectories as in the discrete case.
\end{frame}

\subsection{Real Time Controllers}
\begin{frame}
	\begin{dfn}[Real time Controller]
		A simple real time controller is a function $C: Q \times X \mapsto \Sigma_c \cup {\bot}$
	\end{dfn}

	\medskip

	We denote by $\Sigma_c^\bot = \Sigma_c \cup {\bot}$ the range of controller commands.
	
	\medskip

	We also require that the controller is $k$-polyhedral, i.e., for every $\sigma \in \Sigma_c^\bot, C^{-1}(\sigma)$ is  a $k$-polyhedral set.

	\note{
		According to this function the controller chooses at any configuration $(q,x)$ whether to issue some enabled transition $\sigma$ or to do nothing and let time go by.
		\\
		$\bot$ equals bot
		\\
		$C^{-1}(\sigma)$ means that the domain of $C$ has to be a polyhedral set. We will se later that this conditions is required in the proof.
	} 
\end{frame}

\begin{frame}
	\begin{dfn}[Controlled Trajectories]
		Given a simple controller $C$, a pair $\confPair$ of configurations is a $C$-step if it is either:
		\begin{itemize}
			\item an $e-step$ 
			\item a $\sigma-step$ such that $C(q,x)=\sigma \in \Sigma_c$
			\item a $t-step$ for some $t \in T$ such that for every $t'$, $t' \in [0,t), C(q,x+\textbf{1}t')=\bot$
		\end{itemize}
	\end{dfn}
	A $C$-trajectory is a trajectory consisting of $C$-steps. We denote the set of $C$-trajectories of $\mathcal{T}$ by $L_C(\mathcal{T})$.
\end{frame}

\begin{frame}
	\begin{dfn}[Real time Controller Synthesis]
		Given a timed automaton $\mathcal{T}$ an a acceptance condition $\Omega$, the problem \textbf{RT-Synth}$(\mathcal{T},\Omega)$ is: Construct a real-time controller $C$ such that $L_C(\mathcal{T})\subseteq L(\mathcal{T},\Omega)$
	\end{dfn}
\end{frame}

\subsection{Control Synthesis for Timed Systems}
\begin{frame}
	\begin{dfn}[$(t,\sigma)-successor$]
		For $t \in T$ and $\sigma \in \Sigma$, the configuration $\confP$ is defined to be a $(t,\sigma)-successor$ of the configuration $\conf$ if there exists an intermediate configuration $\confH$ such that $\confH$ is a $t- successor$ of $\conf$ and $\confP$ is a $\sigma-successor$ of $\confH$.
	\end{dfn}
	
	Then we define a function $\delta : (Q \times  X) \times (T \times \Sigma^\bot_c) \mapsto 2^{Q \times X}$ where $\delta(\conf,(t,\sigma))$ stands for all the possible consequences of the controller attempting to issue the command $\sigma \in \Sigma^\bot_c$ after waiting $t$ time units starting at configuration $\conf$
	
	\note{
		In order to tackle the real time controller synthesis problem we introduce the following definitions:\\

		Note that this covers the case of $\confP$ being simply a $\sigma-successor$ 
		of $\conf$ by viewing it as a $(0,\sigma)-successor$ of $\conf$.
	}
\end{frame}

\begin{frame}
	\begin{dfn}[Extended Transition Function]
		For every $t \in T$ and $\sigma \in \Sigma_c$, the set $\delta(\conf,(t,\sigma))$ consists of all the configurations $\confP$ such that:
		\begin{itemize}
			\item $\confP$ is a $(t,\sigma)-successor$ of $\conf$
			\item $\confP$ is a $(t,e)-successor$ of $\conf$ for some $t' \in [0,t]$
		\end{itemize}
	\end{dfn}
	\note{
		This definition covers successor configurations that are obtained
		in one of two possible ways:\\
		some configurations result from the plant waiting patiently at state 
		q for t time units, and then taking a $\sigma$-labeled transition 
		according to the controller recommendation, \\
		the second possibility is of configurations obtained by taking an 
		environment transition at any time $t' \leq t$\\
		This is in fact the crucial new feature of real-time games - there 
		are no turns and the adversary need not wait for the player's next move.
	}
\end{frame}

\begin{frame}
	\begin{dfn}[Controllable Predecessors]
		The controllable predecessors function $\pi:2^Q \times 2^X \mapsto 2^Q \times 2^X$ is defined for every $K \subseteq Q \times X$ by
		$$\pi(K) =\{\conf : \exists t \in T \exists \sigma \in \Sigma_c\;\;\delta(\conf,(t,\sigma)) \subseteq K\}$$
	\end{dfn}
	\note{
		As in the discrete case, we define a predecessor function that indicates the
		configurations from which the controller can force the automaton into a given
		set of configurations.\\
		As in the discrete case, the sets of winning configurations can be characterized
		by a fixed point expressions similar to the discrete one over $2^Q \times 2^X$.
	}
\end{frame}

\begin{frame}
	Assume that $Q = \{q_0,\ldots,q_m\}$. Clearly, any set of configurations ca be written as $ K=\{q_0\} \times P_0 \cup \ldots \cup \{q_m\} \times P_m$ where $P_0,\dots P_m$ are subsets of $X$.

	\medskip

	Thus the set $K$ can be uniquely represented by a set tuple $\mathcal{H} = \langle P_0, \ldots, P_m\rangle$ and we can view $\pi$ as a transformation on set tuples.
\end{frame}

\begin{frame}
	\begin{thm}[Closure of $\mathcal{H}^*_k$ under $\pi$]
		if $\mathcal{H} = \langle P_0, \ldots, P_m\rangle$ is k-polyhedral so is $\pi(\mathcal{H}) = \langle P_0, \ldots, P_m\rangle$
	\end{thm}
\end{frame}

\begin{frame}
	\textbf{Sketch of Proof}
	
	\medskip

	A set tuple $\mathcal{H}$ is called k-polyhedral if each component $P_0,\ldots,P_m$ belongs to $\mathcal{H}^*_k$.

	\medskip

	Wlog, we assume that for every $q \in Q,\sigma \in \Sigma_c$ there is at most one $r=\langle q,\sigma,g,f,q'\rangle \in R$. Let $\langle P_0', \ldots, P_m'\rangle = \pi(\langle P_0, \ldots, P_m\rangle)$.
	
	\medskip

	Then, for each $i=0,\ldots,m$ then set $P_i'$ can be expressed as:

	\noindent 
	\begin{adjustbox}{max width=\textwidth}
	\parbox{\linewidth}{%
		\begin{align*}
			P_i' = \bigcup_{\langle q_i,\sigma,g,f,q_j\rangle \in R} \{x : \exists t \in T \left (
			\begin{aligned}
				\;&x \in I_{q_i} \land\;\;x+\textbf{1}t \in I_{q_i} \land\\ 
				\;&x+\textbf{1}t \in g \land\;\;f(x+\textbf{1}t) \in P_j \land \;\; (\forall t' \leq t)\\
				\;&\bigwedge_{\langle q_i,\sigma,g,f,q_k\rangle \in R}(x+\textbf{1}t' \in g')\rightarrow f(x+\textbf{1}t')\in P_k
			\end{aligned} \right ) \}
		\end{align*}
	}
	\end{adjustbox}
	\note{
		This ugly looking formula just states that $x \in P'_i$ if for some $j, \sigma$ and $t$ 
		we can stay in $q_i$ for $t$ time units and then make a transition to some configuration in
		$\{q_j\}\times P_j$ while all other environment transitions that might be enabled between
		$0$ and $t$ will lead us to a configurations which are in some $\{p_k\} \times P_k$.
	}
\end{frame}

\begin{frame}
	It can be verified that every $P'_i$ can be written as a boolean combinations of sets of the form:

	\begin{equation*}
		\resizebox{.9\hsize}{!}{
			$I_{q_i} \cap \{x : \exists t \in T \;  x+\textbf{1}t \in I_{q_i} \cap g \cap f^{-1}(P_j)\; \forall t' \leq t \; x+\textbf{1}t' \in \overline{g'} \cup f'^{-1}(P_k)\}$
		}
	\end{equation*}
	for some guards $g,g'$ and reset functions $f,f'$, where we use $f^{-1}(P)=\{x : f(x) \in P\}$.

	\medskip
	
	Since timed reachability is distributive over union, i.e.,
	$$\{x:\exists t\;x+\textbf{1}t \in S_1 \cup S_2\} = \{x:\exists t\;x+\textbf{1}t \in S_1\} \cup \{x:\exists t\;x+\textbf{1}t \in S_2\}$$
	it is sufficient to prove the claim assuming $k$-convex polyhedral sets.

	\note{
		The domani of $f^{-1}(P)=\{x : f(x) \in P\}$ is $\mathbb{R}^{+d}$
	}
\end{frame}

\begin{frame}
	So, what remains to show is that for any two $k$-convex sets $S_1$ and $S_2$, the set $\pi_{t',t}(S_1,S_2)$, denoting all the points in $S_1$ from which we can reach $S_2$ without leaving $S_1$, and defined as
	$$\pi_{t',t}(S_1,S_2)=\{x:\exists t \; x+\textbf{1}t\in S_2 \land \forall t'\leq t\;x+\textbf{1}t' \in S_1\}$$
	is also convex.

	\begin{tikzpicture}
		%\draw[step=.5cm,gray,very thin] (-.1,-.1) grid (4.7,4.7);

		\draw[very thick,->] (0,0) -- (4.7,0) node[anchor=north west] {$x_1$};
		\draw[very thick,->] (0,0) -- (0,4.7) node[anchor=south east] {$x_2$};

		%S1
		\draw[thick] (0,0) -- (2.5,0) -- (3.5,1) -- (3.5,3) -- (1,3) -- (0,2) -- cycle;
		\node at (1,1) {$S_1$};

		%S2
		\draw[thick] (2,2) -- (4.5,2) -- (4.5,4.5) -- (3,4.5) -- (2,3.5) -- cycle;
		\node at (4,4) {$S_2$};

		% time
		\draw[thick,->] (1.5,1) -- (2,1.5) node[anchor=north west] {$t$};

		%pi(S1,S2)
		\draw[red,thick,dashed] (0,0) -- (1.5,0) -- (3.5,2) -- (3.5,3) -- (2,3) -- (0,1) -- cycle;
		

		% point that stays in S1
		\draw[green,thick,->] (1.5,1.5) -- (3.5,3.5);

		% point that go out of S1
		\draw[yellow,thick,->] (3,1) -- (4.25,2.25);

		% poit that doesn't reach S2
		\draw[red,thick,->] (0.5,2.25) -- (2.5,4.25);


		\fill[black] (1.5,0) circle (1.5pt);
		\node at (1.5,-.25) {$c'$};

		\fill[black] (4.5,0) circle (1.5pt);
		\node at (4.5,-.25) {$c$};

	\end{tikzpicture}
\end{frame}


\begin{frame}
	\begin{thm}[Control Synthesis for Timed systems]
	Given a timed automaton $\mathcal{T}$ and an acceptance condition
	$$\AccpCond$$
	the problem \textbf{RT-Synth}$(\mathcal{T},\Omega)$ is solvable
	\end{thm}
\end{frame}


\begin{frame}
	\textbf{Sketch of Proof}
	
	\medskip
	
	We have just shown that $2^Q \times \mathcal{H}^*_k$ is closed under $\pi$.
	
	\medskip
	
	Any of the iterative processes for the fixed point equations $(1)-(5)$ starts with an
	element of $2^Q \times \mathcal{H}^*_k$.
		
	\medskip
	
	For example, the iteration for $\Diamond$ starts with $W_0=Q\times F$.
		
	\medskip
	
	Each iteration consists of applying Boolean set-theoretic operations and the predecessor operation, 
	which implies that every $W_i$ is also an element of $2^Q \times \mathcal{H}^*_k$ - a
	finite set. 
		
	\medskip
	
	Thus, by monotonicity, a fixed point is eventually reached.
	\note{
		The strategy is extracted in a similar manner as in the discrete case. When
		ever a configuration $\conf$ is added to $W$, it is due to one or more pairs of the
		form $([t_1,t_2],\sigma)$ indicating that within any $t,t_1<t<t_2$ issuing $\sigma$ after waiting
		$t$ will lead to a winning position. Hence by letting $C\conf = \bot$ when $t_1>0$ and
		$C\conf = \sigma$ when $t_1=0$ we obtain a $k-$polyhedral controller.
	}
\end{frame}

\section{Citations}
\begin{frame}
	\bibliography{bibliography}
\end{frame} 

\end{document}
