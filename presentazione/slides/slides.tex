\documentclass[table]{beamer}

\usepackage[utf8]{inputenc}

\usepackage[default]{droidserif}
\usepackage[T1]{fontenc}
\usepackage{ragged2e}

\usepackage{changepage}


% per immagini
\usepackage{graphicx}
\graphicspath{ {./images/} }

% per testo colotaro
\usepackage{xcolor}

% colori utilizzati
\definecolor{marrone}{RGB}{92,64,60}
\definecolor{arancio}{RGB}{255,102,0}
\definecolor{azzurro}{RGB}{119,154,171}
\definecolor{azzurro_chiaro}{RGB}{198,214,220}
\definecolor{bianco}{RGB}{255,255,255}


%--------------------------------------------------------------------------------
%						BLOCCO PER TEOREMI E DEFINIZIONI
%--------------------------------------------------------------------------------
\usepackage{amsthm}
\usepackage{thmtools}
\usepackage{thmbox}

\makeatletter
\def\thmbox@color{black}
\define@key{thmbox}{color}{\def\thmbox@color{#1}}
\def\thmbox@head#1{%
  \par\noindent\vbox{%
    \setbox\thmbox@box@=\hbox{%
      \vrule width 0mm height 0mm depth \thmbox@vskip%
      #1}%
    \copy\thmbox@box@%
    \ifthmbox@underline%
       \color@begingroup\color{\thmbox@color}\hrule width \wd\thmbox@box@ height \thmbox@thickness\color@endgroup%
    \fi}%
  \hrule height 0mm\relax}
\def\thmbox@put#1{
  \vskip\z@%
  \noindent%
  \hbox{%
    {\dimen0=\thmbox@leftmargin%
     \advance\dimen0-\thmbox@hskip%
     \advance\dimen0-\thmbox@thickness%
     \hskip\dimen0}%
    \color@begingroup\color{\thmbox@color}\vrule width \thmbox@thickness\color@endgroup%
    \hskip\thmbox@hskip%
    \box#1%
    \ifx\thmbox@style L%
      \hskip\thmbox@hskip%
      \color@begingroup\color{\thmbox@color}\vrule width \thmbox@thickness\color@endgroup%
    \fi}%
  \par\nobreak}
\def\thmbox@tail{%
  \hrule height 0mm%
  \ifx\thmbox@style M%
    \thmbox@dim=1cm%
  \else\ifx\thmbox@style L%
    \thmbox@dim=\hsize%
    \advance\thmbox@dim-\thmbox@leftmargin%
    \advance\thmbox@dim-\thmbox@rightmargin%
    \advance\thmbox@dim2\thmbox@hskip%
    \advance\thmbox@dim2\thmbox@thickness%
  \fi\fi%
  \noindent%
  {\dimen0=\thmbox@leftmargin%
   \advance\dimen0-\thmbox@hskip%
   \advance\dimen0-\thmbox@thickness%
   \hskip\dimen0}%
  \color@begingroup\color{\thmbox@color}\vrule width \thmbox@dim height \thmbox@thickness\color@endgroup%
  \par}
\makeatother

%\declaretheorem[shaded={rulecolor=azzurro,rulewidth=2pt, bgcolor={rgb}{1,1,1}}]{defn}{Definition}
%\declaretheorem[thmbox=M]{Definition}
\newtheorem[style=L, thickness=1pt,color=azzurro]{thm}{Theorem}
\newtheorem[style=L, thickness=1pt,color=arancio]{dfn}{Definition}
%--------------------------------------------------------------------------------
%						BLOCCO PER TEOREMI E DEFINIZIONI
%--------------------------------------------------------------------------------



%--------------------------------------------------------------------------------
%						BLOCCO PER PRESENTAZIONI UNIUD
%--------------------------------------------------------------------------------
\setbeamercolor{marrone}{fg=white, bg=marrone}
\setbeamercolor{arancio}{fg=white, bg=arancio}
\setbeamercolor{azzurro}{fg=white, bg=azzurro}

\setbeamertemplate{headline}{%
	\leavevmode%
	\begin{beamercolorbox}[sep=0.15cm,ht=1cm,wd=1.05cm]{marrone}%
		\includegraphics[width=.75cm]{UdineLogoFull}
	\end{beamercolorbox}%
	\begin{beamercolorbox}[sep=0.15cm,ht=1cm,wd=.4\paperwidth]{arancio}%
		\textbf{UNIVERSITÀ \linebreak
		DEGLI STUDI \linebreak
		DI UDINE}
	\end{beamercolorbox}%
	\begin{beamercolorbox}[ht=1cm,wd=.6\paperwidth]{azzurro}%
	\end{beamercolorbox}%
}

\setbeamertemplate{frametitle}{
	\begin{flushright}
		\vspace*{-1.25cm}\color{bianco}
		\insertframetitle
	\end{flushright}
}

\logo{\includegraphics[scale=.5]{UdineLogo}\hspace*{.65\paperwidth}}
%--------------------------------------------------------------------------------
%					FINE BLOCCO PER PRESENTAZIONI UNIUD
%--------------------------------------------------------------------------------


%Information to be included in the title page:
\title[About Beamer] %optional
{About the Beamer class in presentation making}

\subtitle{A short story}

\author[Enrico] % (optional, for multiple authors)
{E. Cominato 137396\inst{1}}

\institute % (optional)
{
	\inst{1}%
	Dipartimento di Scienze Matematiche, Informatiche e Fisiche\\
	Università degli studi di Udine
}

\date[VLC 2013] % (optional)
{Very Large Conference, April 2013}



\begin{document}

\frame{\titlepage}

\section{Initial Definitions}
\newcommand{\Plant}{\ensuremath{\mathcal{P}=(Q,\Sigma_c,\delta, q)}}
\newcommand{\Controller}{\ensuremath{C:Q^+\longmapsto \Sigma_c}}
\newcommand{\AccpCond}{\ensuremath{\{(F,\square),(F,\Diamond ),(F,\Diamond\square),(F,\square\Diamond),(\mathcal{F},\mathcal{R}_n)\}}}
\newcommand{\Synth}{\ensuremath{\textbf{Synth}(\mathcal{P},\Omega)}}
\begin{frame}
	\frametitle{Discrete case}
	\begin{dfn}[Plant]
		A plant automaton is a tuple $\Plant$ where
		$Q$ is a finite set of states, $\Sigma_c$ is a set of controller commands, 
		$\delta:Q \times \Sigma_c \longmapsto 2^Q$ is the transition function and 
		$q_o \in Q$ is an initial state.
	\end{dfn}

	\begin{dfn}[Controllers]
		A controller (strategy) for a plant specified by $\Plant$ 
		is a function $\Controller$. A simple controller is a controller that 
		can be written as a function $C:Q \longmapsto \Sigma_c$.
	\end{dfn}
	We are interested in the simpler cases of controllers that base their decisions on a finite memory.
\end{frame}

\begin{frame}
	\frametitle{Discrete case}
	\begin{dfn}[Trajectories]
		Let $\mathcal{P}$ be a plant and let $\Controller$ be a controller. 
		An infinite sequence of states $\alpha:q[0],q[1],\ldots$ such that
		 $q[0]=q_0$ is called a trajectory of $\mathcal{P}$ if 
		$$q[i+1] \in \bigcup_{\sigma \in \Sigma_c}\delta(q[i],\sigma)$$
		and a C-trajectory if $q[i+1] \in \delta(q[i],C[\alpha[0..i]])$ for every $i\geq 0$.
		The corresponding sets of trajectories are denoted by $L(\mathcal{P})$ and $L_C(\mathcal{P})$.
	\end{dfn}
\end{frame}

\begin{frame}
	For every infinite trajectory $\alpha \in L(\mathcal{P})$:
	\begin{itemize}
		\item $Vis(\alpha)$ denote the set of all states appearing in $\alpha$
		\item $Inf(\alpha)$ denote the set of all states appearing in $\alpha$ infinitely many times
	\end{itemize}
\end{frame}

\begin{frame}
	\begin{dfn}[Acceptance Condition]
		Let $\Plant$ be a plant. An acceptance condition for $\mathcal{P}$ is
		$$ \Omega \in \AccpCond$$
		where $\mathcal{F} = \{(F_i,G_i)\}^{n}_{i=1}$ and $F, F_i$ and $G_i$ 
		are certain subsets of $Q$ referred as the good states. The set of sequences of $\mathcal{P}$ that are accepted
		accordig to $\Omega$ is defined as follows:
	\end{dfn}
	\begin{adjustwidth}{-1cm}{-1cm}
	\begin{table}[]
		\begin{tabular}{lll}
		\hline
		\rowcolor{azzurro_chiaro}
		$L(\mathcal{P},F,\square)$ & $\{\alpha \in L(\mathcal{P}):Vis(\alpha)\subseteq F\}$ & $\alpha$ always remains in $F$ \\ 
		$L(\mathcal{P},F,\Diamond)$ & $\{\alpha \in L(\mathcal{P}):Vis(\alpha)\cap F \neq \emptyset\}$ & $\alpha$ eventually visits $F$ \\ 
		\rowcolor{azzurro_chiaro}
		$L(\mathcal{P},F,\Diamond\square)$ & $\{\alpha \in L(\mathcal{P}):Inf(\alpha)\subseteq F\}$ & $\alpha$ eventually remains in $F$ \\ 
		$L(\mathcal{P},F,\square\Diamond)$ & $\{\alpha \in L(\mathcal{P}):Inf(\alpha)\cap F \neq \emptyset\}$ & $\alpha$ visits $F$ infinitely often \\ 
		\rowcolor{azzurro_chiaro}
		$L(\mathcal{P},\mathcal{F},\mathcal{R}_n)$ & \shortstack{
			$\{\alpha \in L(\mathcal{P}): \exists i \alpha \in$ \\
			$L(\mathcal{P},F_i,\square\Diamond) \cap L(\mathcal{P},G_i,\Diamond\square)\}$
		} & 
		\shortstack{
			$\alpha$ visits $F_i$ infinitely often \\
			 and eventually stays in $G_i$ }\\ \hline
		\end{tabular}
	\end{table}
	\end{adjustwidth}
\end{frame}

\begin{frame}
	\frametitle{Discrete case}
	\begin{dfn}[Controller Synthesis Problem]
		For a plant $\mathcal{P}$ and an acceptance condition $\Omega$, the problem $\Synth$ is:
		Find a controller C such that $L_C(\mathcal{P})\subseteq L(\mathcal{P},\Omega)$ ot otherwise show that such
		 a controller does not exists.
	\end{dfn}
\end{frame}

\begin{frame}
	\begin{dfn}[Controllable Predecessors]
		Let $\mathcal{P}=(Q,\Sigma_c,\delta, q)$ be a plant and a set of states $P \subseteq Q$. 
		The controllable predecessors of $P$ is the set of states from which the controller 
		can "force" the plant into P in one step:
		$$ \{q : \exists \sigma \in \Sigma_c \centerdot  \delta(q,\sigma) \subseteq P\} $$
	\end{dfn}
	We define a function $\pi:2^Q \longmapsto 2^Q$, mapping a set of states $P \subseteq Q$ into 
	the set of its Controllable predecessors:
	$$ \pi(P) = \{q : \exists \sigma \in \Sigma_c \centerdot  \delta(q,\sigma) \subseteq P\} $$
\end{frame}


\begin{frame}
	\begin{center}
		\begin{tikzpicture}[shorten >=1pt,node distance=2.5cm,on grid,auto] 
			\node[state] (q_0)					{};
			\node[state] (q_1)	[right=of q_0]	{};
			\node[state] (q_2)	[right=of q_1]	{};

			% stati predecessori
			\node[state] (p_0)	[above=of q_0]	{};
			\node[state] (p_1)	[above=of q_1]	{};
			\node[state] (p_2)	[above=of q_2]	{};
			\node[state] (p_3)	[right=of p_2]	{};
			
			% raggruppamenti
			\begin{scope}[on background layer]
				\node[fit=(q_0) (q_1) (q_2), ellipse, 
				draw=arancio,fill=arancio!30, draw=arancio, fill opacity=0.4, label=below:P] (P) {};
				\node[fit=(p_0) (p_1) (p_2), ellipse, 
				draw=azzurro, fill=azzurro!30, draw=azzurro, fill opacity=0.4, label=above:$\pi(P)$] (predP) {};
			\end{scope}

			\path[->] 
				(p_0) edge  node {} (q_0)
				(p_1) edge  node {} (q_0)
				(p_1) edge  node {} (p_0)
					  edge  node {} (q_1)
				(p_2) edge  node {} (q_2)
				(p_3) edge  node {} (p_2);
		\end{tikzpicture}
	\end{center}
\end{frame}

\begin{frame}
	\frametitle{Discrete case}
	\begin{thm}
		For every $\Omega \in \AccpCond$
		the problem $\Synth$ is solvable. Moreover, if $(\mathcal{P},\Omega)$ is controllable then 
		it is controllable by a simple controller.
	\end{thm}
\end{frame}

\begin{frame}
	\frametitle{Discrete case}
	\begin{dfn}[Winning states]
		For a plant $\Plant$ and an acceptance condition $\Omega$, we denote $W \subseteq Q$ as the set
		of winning states, namely, the set of states from which a controller can enforce good behaviors according to $\Omega$.
	\end{dfn}
\end{frame}
\begin{frame}
	We can characterize this states by the following fixed-point expressions:
	\begin{itemize}
		\item[$\square$] $\nu W \big( F \cap \pi(H)\big)$ 
		\item[$\Diamond$] $\nu W\big(F \cup \pi(W)\big)$ 
		\item[$\Diamond\square$] $\mu W \nu H\Big(\pi(H) \cap \big(F \cup \pi(W)\big)\Big)$
		\item[$\square\Diamond$] $\nu W \mu H\Big(\pi(H) \cup \big(F \cap \pi(W)\big)\Big)$
		\item[$\mathcal{R}_1$] $\mu W \Bigg\{\pi(W) \cap \nu Y \mu H . W \cup G \cap \Big(\pi(H) \cup \big(F \cap \pi(Y)\big)\Big)\Bigg\}$
	\end{itemize}
	Then the plant is controllable iff $q_0 \in W$
\end{frame}
\begin{frame}
	Let see in more details how this works. Consider the case $\Diamond$:
	\begin{table}[]
		\begin{tabular}{lll}
		$\begin{aligned}
			&W_0 := \emptyset \\
			&\textrm{\textbf{for} } i:=0,1,\ldots \textrm{ \textbf{repeat}} \\
			& \quad W_{i+1} := F \cup \pi(W_i) \\
			&\textrm{\textbf{until} } W_{i+1} = W_i\\
		\end{aligned}$
		& $\quad$ &
		$\begin{aligned}
			&W_0 := \emptyset \\
			&W_1 := F \cup \pi(W_0) = F \cup \pi(W_0) = F \\
			&W_2 := F \cup \pi(W_1) = F \cup \pi(F)\\
			& \ldots \\
		\end{aligned}$ \\
		&&\\% empty line
		\multicolumn{3}{l}{finally: $W_n := F \cup \pi(W_{n-1}) = F \cup \pi(F \cup \pi(\ldots(F \cup \pi(F))))$}
		\end{tabular}
	\end{table}
\end{frame}
\begin{frame}
	\justify
	In the process of calculating $W_i+1$, whenever we add a state $q$ to $W_i$, 
	there must be at least one action $\sigma \in \Sigma_c$ such that $\delta(q,\sigma) \subseteq W_i$.
	
	\medskip	
	
	So we define the controller at $q$ as $C(q)=\sigma$.
	
	\medskip
	
	When the process terminates, the controller is synthesized for all the winning states.
		
	\medskip
	
	It can be seen that if the process fails, that is, $q_0 \not\in W$ then for every controller
	command there is a possibly bad consequence that will put the system outside $F$, 
	and no controller, even an infinite state one, can prevent this.
\end{frame}
\begin{frame}
	Consider now the case $\square\Diamond$:
	\begin{table}[]
		\begin{tabular}{lll}
		$\begin{aligned}
			&W_0 := Q \\
			&\textrm{\textbf{for} } i:=0,1,\ldots \textrm{ \textbf{repeat}} \\
			&\quad H_0 := \emptyset \\
			&\quad \textrm{\textbf{for} } j:=0,1,\ldots \textrm{ \textbf{repeat}} \\
			&\quad \quad H_{j+1} := \pi(H_j) \cup (F \cap \pi(W_i)) \\
			&\quad \textrm{\textbf{until} }H_{j+1} = H_j\\
			& \quad W_{i+1} := H_j \\
			&\textrm{\textbf{until} } W_{i+1} = W_i\\
		\end{aligned}$
		& $\quad$ &
		$\begin{aligned}
			&W_0 := Q & H_0 := \emptyset \\
			&W_0 := Q & H_1 := F \cap \pi(Q) \\
			& \\
			& W_1 := F \cap \pi(Q)
			&W_2 := F \cup \pi(W_1) = F \cup \pi(F)\\
			& \ldots \\
		\end{aligned}$ \\
		&&\\% empty line
		\multicolumn{3}{l}{finally: $W_n := F \cup \pi(W_{n-1}) = F \cup \pi(F \cup \pi(\ldots(F \cup \pi(F))))$}
		\end{tabular}
	\end{table}
\end{frame}
\end{document}