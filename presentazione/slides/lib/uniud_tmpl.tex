
% per immagini
\usepackage{graphicx}
\graphicspath{ {./images/} }

% per testo colotaro
\usepackage{xcolor}

% colori utilizzati
\definecolor{marrone}{RGB}{92,64,60}
\definecolor{arancio}{RGB}{255,102,0}
\definecolor{azzurro}{RGB}{119,154,171}
\definecolor{azzurro_chiaro}{RGB}{198,214,220}
\definecolor{bianco}{RGB}{255,255,255}

% per i grafi
\usepackage{tikz}
\usetikzlibrary{automata,positioning,fit,shapes.geometric,backgrounds}
% per annotazioni sui grafi
\usetikzlibrary{decorations.pathreplacing,angles,quotes}

%--------------------------------------------------------------------------------
%						BLOCCO PER TEOREMI E DEFINIZIONI
%--------------------------------------------------------------------------------
\usepackage{amsthm}
\usepackage{thmtools}
\usepackage{thmbox}

\makeatletter
\def\thmbox@color{black}
\define@key{thmbox}{color}{\def\thmbox@color{#1}}
\def\thmbox@head#1{%
  \par\noindent\vbox{%
    \setbox\thmbox@box@=\hbox{%
      \vrule width 0mm height 0mm depth \thmbox@vskip%
      #1}%
    \copy\thmbox@box@%
    \ifthmbox@underline%
       \color@begingroup\color{\thmbox@color}\hrule width \wd\thmbox@box@ height \thmbox@thickness\color@endgroup%
    \fi}%
  \hrule height 0mm\relax}
\def\thmbox@put#1{
  \vskip\z@%
  \noindent%
  \hbox{%
    {\dimen0=\thmbox@leftmargin%
     \advance\dimen0-\thmbox@hskip%
     \advance\dimen0-\thmbox@thickness%
     \hskip\dimen0}%
    \color@begingroup\color{\thmbox@color}\vrule width \thmbox@thickness\color@endgroup%
    \hskip\thmbox@hskip%
    \box#1%
    \ifx\thmbox@style L%
      \hskip\thmbox@hskip%
      \color@begingroup\color{\thmbox@color}\vrule width \thmbox@thickness\color@endgroup%
    \fi}%
  \par\nobreak}
\def\thmbox@tail{%
  \hrule height 0mm%
  \ifx\thmbox@style M%
    \thmbox@dim=1cm%
  \else\ifx\thmbox@style L%
    \thmbox@dim=\hsize%
    \advance\thmbox@dim-\thmbox@leftmargin%
    \advance\thmbox@dim-\thmbox@rightmargin%
    \advance\thmbox@dim2\thmbox@hskip%
    \advance\thmbox@dim2\thmbox@thickness%
  \fi\fi%
  \noindent%
  {\dimen0=\thmbox@leftmargin%
   \advance\dimen0-\thmbox@hskip%
   \advance\dimen0-\thmbox@thickness%
   \hskip\dimen0}%
  \color@begingroup\color{\thmbox@color}\vrule width \thmbox@dim height \thmbox@thickness\color@endgroup%
  \par}
\makeatother

%\declaretheorem[shaded={rulecolor=azzurro,rulewidth=2pt, bgcolor={rgb}{1,1,1}}]{defn}{Definition}
%\declaretheorem[thmbox=M]{Definition}
\newtheorem[style=L, thickness=1pt,color=azzurro]{thm}{Theorem}
\newtheorem[style=L, thickness=1pt,color=arancio]{dfn}{Definition}
%--------------------------------------------------------------------------------
%						BLOCCO PER TEOREMI E DEFINIZIONI
%--------------------------------------------------------------------------------



%--------------------------------------------------------------------------------
%						BLOCCO PER PRESENTAZIONI UNIUD
%--------------------------------------------------------------------------------
\setbeamercolor{marrone}{fg=white, bg=marrone}
\setbeamercolor{arancio}{fg=white, bg=arancio}
\setbeamercolor{azzurro}{fg=white, bg=azzurro}

\setbeamertemplate{headline}{%
	\leavevmode%
	\begin{beamercolorbox}[sep=0.15cm,ht=1cm,wd=.1\paperwidth]{marrone}%
		\includegraphics[width=.75cm]{UdineLogoFull}
	\end{beamercolorbox}%
	\begin{beamercolorbox}[sep=0.15cm,ht=1cm,wd=.3\paperwidth]{arancio}%
		\textbf{UNIVERSITÀ \linebreak
		DEGLI STUDI \linebreak
		DI UDINE}
	\end{beamercolorbox}%
	\begin{beamercolorbox}[ht=1cm,wd=.6\paperwidth]{azzurro}%
		\vspace*{\fill}
			\color{bianco}
			{\huge \insertsectionhead}
		\vspace*{\fill}	
	\end{beamercolorbox}%
}

\setbeamertemplate{frametitle}{
	\begin{flushright}
		\vspace*{-1.25cm}\color{bianco}
		\insertframetitle
	\end{flushright}
}

\logo{\includegraphics[scale=.5]{UdineLogo}\hspace*{.65\paperwidth}}
%--------------------------------------------------------------------------------
%					FINE BLOCCO PER PRESENTAZIONI UNIUD
%--------------------------------------------------------------------------------


\makeatletter

\pgfkeys{/pgf/.cd,
    ellipse ratio/.code={\pgfkeyssetvalue{/pgf/ellipse ratio}{#1}},
    ellipse ratio/.initial=1
}

\pgfdeclareshape{newellipse}
{
  \inheritsavedanchors[from=ellipse]
  \inheritanchorborder[from=ellipse]
  \savedanchor\radius{%
    % 
    % Caculate ``height radius''
    % 
    \pgf@y=.5\ht\pgfnodeparttextbox%
    \advance\pgf@y by.5\dp\pgfnodeparttextbox%
    \pgfmathsetlength\pgf@yb{\pgfkeysvalueof{/pgf/inner ysep}}%
    \advance\pgf@y by\pgf@yb%
    % 
    % Caculate ``width radius''
    % 
    \pgf@x=.5\wd\pgfnodeparttextbox%
    \pgfmathsetlength\pgf@xb{\pgfkeysvalueof{/pgf/inner xsep}}%
    \advance\pgf@x by\pgf@xb%
    % 
    % Adjust
    %
    \pgfkeysgetvalue{/pgf/ellipse ratio}{\ratioscale}
    \pgfmathsetmacro\widthfactor{sqrt(\ratioscale^2+1)/\ratioscale}
    \pgfmathsetmacro\heightfactor{sqrt(\ratioscale^2+1)}
    \pgf@x=\widthfactor\pgf@x%
    \pgf@y=\heightfactor\pgf@y%
    % 
    % Adjust height, if necessary
    % 
    \pgfmathsetlength\pgf@yc{\pgfkeysvalueof{/pgf/minimum height}}%
    \ifdim\pgf@y<.5\pgf@yc%
      \pgf@y=.5\pgf@yc%
    \fi%
    % 
    % Adjust width, if necessary
    % 
    \pgfmathsetlength\pgf@xc{\pgfkeysvalueof{/pgf/minimum width}}%
    \ifdim\pgf@x<.5\pgf@xc%
      \pgf@x=.5\pgf@xc%
    \fi%
    % 
    % Add outer sep
    % 
    \pgfmathsetlength{\pgf@xb}{\pgfkeysvalueof{/pgf/outer xsep}}%  
    \pgfmathsetlength{\pgf@yb}{\pgfkeysvalueof{/pgf/outer ysep}}%  
    \advance\pgf@x by\pgf@xb%
    \advance\pgf@y by\pgf@yb%
  }

  \inheritanchor[from=ellipse]{center}
  \inheritanchor[from=ellipse]{mid}
  \inheritanchor[from=ellipse]{base}
  \inheritanchor[from=ellipse]{north}
  \inheritanchor[from=ellipse]{south}
  \inheritanchor[from=ellipse]{west}
  \inheritanchor[from=ellipse]{mid west}
  \inheritanchor[from=ellipse]{base west}
  \inheritanchor[from=ellipse]{north west}
  \inheritanchor[from=ellipse]{south west}
  \inheritanchor[from=ellipse]{east}
  \inheritanchor[from=ellipse]{mid east}
  \inheritanchor[from=ellipse]{base east}
  \inheritanchor[from=ellipse]{north east}
  \inheritanchor[from=ellipse]{south east}

  \inheritbackgroundpath[from=ellipse]
}
\makeatother